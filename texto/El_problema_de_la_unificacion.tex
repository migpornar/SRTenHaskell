\chapter{El problema de la unificación y normalización de términos}

En este capítulo vamos definir los problemas de decisión en el
razonamiento ecuacional. Nos centraremos en los sistemas de
reescritura de términos y la unificación.  Para ello consideraremos la
siguiente definición.

\begin{defi}
  Dado un conjunto de identidades $E$ y una ecuación $s \approx t$, diremos que la ecuación es, \\
  $\begin{array}{llll}
    $\textbf{Válida}$        &  $ en $E$ syss s $s \approx_E t  \\
    $\textbf{Satisfactible}$  &  $ en $E$ syss hay una sustitución $\sigma$ tal que $\sigma(s) \approx_E \sigma(t) \\
   \end{array} 
  $
\end{defi}

En la siguiente sección obtendremos varias propiedades interesantes sobre $\approx_E$.

\section{Estudio de $\approx_E$}

En el teorema \ref{teor:conflu}, vimos que $\xleftrightarrow{*}$ era
decidible si $\rightarrow_E$ si es convergente. De esta manera
podíamos computar $\downarrow_E$. Por ello es importante saber si
podemos decidir si un término implica a otro término. Esto se llama el
problema del emparejamiento, dado dos términos $s$ y $l$, ver si
existe una sustitución $\sigma(l) = s$, y computar $\sigma$ si
existe. 

Comprobaremos que el problema del emparejamiento y de la
satisfactibilidad están muy relacionados.

\begin{teor} \label{teor:3.1}
  Si $E$ es finito y $\rightarrow_E$ es convergente, entonces
  $\approx_E$ es decidible.
\end{teor}

\begin{demo}
  Por el teorema \ref{teor:conflu}, sabemos que $s \xleftrightarrow{*}_E t$ syss
  $s\downarrow_E = t\downarrow_E$.  La forma normal del operador
  $\downarrow_E$ es computable, pues podemos decidir si un termino $u$
  es forma normal, y si no lo fuese, podemos computar un $u'$ tal que
  $u \rightarrow_E u'$, y $u'$ fuese una forma normal.
	
  Para ver si $u$ es una forma normal, tenemos que comprobar para
  todas las identidades de $E$ y todas la posiciones $p \in Pos(u)$ si
  hay una sustitución $\sigma$ tal que $u|_p = \sigma l$. Como el
  problema del emparejamiento es decidible y $E$ es finito, $u$ es una
  forma normal o podemos reducir $u$ en un proceso iterativo finito a
  una formal normal. Este proceso termina pues $\rightarrow_E$ es
  terminador.
\end{demo}

Por este resultado, surge un interés en las reducciones convergentes.
El problema de la palabra para $E$ trata de decidir $s \approx t$ para
arbitrarios $s,t, \in T(\Sigma, V)$. También tenemos el problema de
las palabras básicas para $E$, que no es más que el problema de la
palabra restringido a términos básicos $s,t$.

\begin{ejem}
  Considerando la signatura $\Sigma = \{*, S, K\}$ donde $*$ es una
  función binaria y $S$ y $K$ son constantes. El conjunto de
  identidades considerado es,

  \[E := \{((S \cdot x) \cdot y) \cdot z \approx (x \cdot z) \cdot (y
    \cdot z),(K \cdot x) \cdot y \approx x \} \]

  Como se puede dar cualquier término a partir de uno básico, esto
  implica que el problema de la palabra para E sea indecidible.
\end{ejem}

\section{Sistemas de reescritura de términos}

En esta sección vamos a introducir la definición de regla y sistema de
reescritura, así como una propiedad.

\begin{defi}
  Una regla de reescritura es una identidad $l \approx r$ tal que $l$
  no es una variable y $\Var(l) \supseteq \Var(r)$. En ese caso,
  escribiremos $l \rightarrow r$. Un sistema de reescritura de
  términos (SRT) es un conjunto de reglas de escritura. Una expresión
  reducible es una instancia del lado derecho de una regla de
  reescritura. Contraer la expresión reducible significa sustituirla
  por la instancia del lado derecho de la regla.
\end{defi}

Como cualquier SRT es un conjunto de identidades, $\rightarrow_R$ está
bien definido.

A partir del teorema \ref{teor:3.1}, deducimos el siguiente resultado,

\begin{teor}
  Si $R$ es un SRT convergente finito, $\approx_R$ es decidible: $s
  \approx_R t$ syss $s \downarrow_R = t \downarrow_R$.
\end{teor}

Veamos ahora una de las propiedades básicas de $\rightarrow_R$.

\begin{defi}
  Una relación en $T(\Sigma, V)$ es una relación de reescritura syss
  es compatible con las $\Sigma$-operaciones y cerrada bajo
  sustitución.
\end{defi}

Por el \ref{lema:2.2}, sabemos que $\rightarrow_R$ es una relación de
reescritura. Por inducción de las reglas de derivación,
$\xrightarrow{*}_R$ y $\xrightarrow{+}_R$ también son relaciones de
reescritura. Para $\xleftrightarrow{*}_R$ lo demuestra el teorema \ref{teor:2.1}

\section{Unificación sintáctica}

En esta sección nos centraremos en estudiar la unificación. La
unificación es el proceso de resolver la satisfactibilidad de un
problema. Dado $E$, $s$ y $t$, hay que encontrar una sustitución
$\sigma$ tal que $\sigma s \approx _E \sigma t$. Si $s$ y $t$ son
básicos, el problema pasaría a ser un problema de palabras básicas. En
esta sección estudiaremos el caso especial $E = \emptyset$ que es
importante para algunos algoritmos simbólicos de computación. El que
especialmente nos importa es un algoritmo que comprueba la confluencia
a partir de pares críticos, que veremos mas adelante.

La sustitución o el unificador que busquemos va a
depender del caso en el que nos encontremos. Por ejemplo $x \approx^?
f(y)$ tiene varios unificadores $\{x \mapsto f(y) \}$, $\{x \mapsto
f(a), y \mapsto a\}$. No siempre puede existir uno o varios
unificadores por ejemplo, $f(x) \approx^? g(x)$ no tiene
unificador. Para el caso en el que existan varios unificadores, nos
gustaría encontrar el más general,

\begin{defi}
  Una sustitución $\sigma$ es mas general que una sustitución
  $\sigma'$ si hay una sustitución $\delta$ tal que $ \sigma' = \delta
  \sigma$. En este caso escribimos $\sigma \lesssim \sigma'$. Diremos
  que $\sigma'$ es una instancia de $\sigma$.
\end{defi}

En el ejemplo anterior, si $\sigma = \{x \mapsto f(y) \}$ y
$\sigma' = x \mapsto f(a), y \mapsto a\}$ entonces
$\sigma \lesssim \sigma'$ porque $\sigma' = \delta \sigma$ donde
$\delta = \{ y \mapsto a \}$.

Una propiedad inmediata de la definición es,
%escribir intr de \sim
\begin{lema} \label{lema:3.1}
  $\sigma \sim \sigma '$ syss existe una sustitución inyectiva $p$ que
  sea biyectiva para las variables tal que $\sigma = p \sigma '$
\end{lema}

Como nuestro objetivo es resolver conjuntos de ecuaciones,
formalizaremos el problema de la unificación con la siguiente definición,

\begin{defi}
  Un \textbf{unificador} de un conjunto finito de ecuaciones
  $S = \{ s_1 =^? t_1, \dots, s_n =^? t_n\}$ es una sustitución $\sigma$
  tal que $\sigma s_i = \sigma t_i$ para $i = 1, \dots, n.$ Por $U(S)$
  se denota el conjunto de todos los unificadores de $S$. Se dice que
  $S$ es \textbf{unificable} si $U(S) \neq \emptyset$. El
  \textbf{problema de la unificación} de $S$ consiste en calcular sus
  unificadores.
\end{defi}

Otra definición importante es,

\begin{defi} 
  Una sustitución $\sigma$ es el \textbf{unificador de máxima
    generalidad} (UMG) de $S$ si $\sigma$ es el menor elemento de
  $U(S)$; es decir,
  \begin{enumerate}
  \item $\sigma \in U(s)$
  \item $\forall \sigma' \in U(s) \sigma \lesssim \sigma'$
  \end{enumerate}
\end{defi}

Por ejemplo, $\sigma ' := \{ x \mapsto z, y \mapsto z \}$ es un
unificador de $x=f =^? y$, pero no es UMG, pues es menos general que
$\sigma = {x \mapsto y}$.

\begin{defi}
  Una sustitución $\sigma$ es \textbf{idempotente} si $\sigma = \sigma \sigma$.
\end{defi}

\begin{lema} \label{lema:3.2} Una sustitución $\sigma$ es idempotente
  syss $\Dom(\sigma) \cap \VRan(\sigma) = \emptyset $
\end{lema}

El ultimo resultado de este capítulo se usará en la implementación del
problema de la unificación.

\begin{teor}
  Si un problema de unificación $S$ tiene solución, entonces tiene un
  UMG idempotente.
\end{teor}

\begin{demo} %volver a mirar esta demo
  El lema \ref{lema:3.1} nos dice que los UMG son únicos respecto a
  una sustitución $p$ inyectiva y biyectiva para las variables. Por
  ello, incluso para los idempotentes UMG no serán únicos.
\end{demo}

\section{Unificación por transformación}

La unificación puede ser representada como la aplicación iterada de
transformaciones en un conjunto de ecuaciones. Por ejemplo el
algoritmo que seguimos para resolver sistemas lineales,

% HACER EJEMPLO

\begin{defi}
  El problema de la unificación de 
  $S = \{ x_1 =^? t_1, \dots, x_n =^? t_n \}$ 
  está en su \textbf{forma resuelta}, si las variables $x_i$ son distintas
  entre sí y ninguna $x_i$ ocurre en $t_i$. En este caso definimos, 
  $\vec{S} := \{x_1 \mapsto t_1, \dots, x_n \mapsto t_n \}$.
\end{defi}

Varias propiedades básicas de $\vec{S}$ son,

\begin{lema} \label{lema:3.3}
  Si $S$ está en un su forma resuelta entonces, 
  $\sigma = \sigma \vec{S}$ 
  para todo $\sigma \in U(S)$.
\end{lema}

\begin{demo}
  Sea $S = \{ x_1 =^? t_1, \dots, x_n =^? t_n \}$. Probaremos por
  casos que $\forall x \in V. \sigma x = \sigma \vec{S} x$

  Para el caso $x \in \{x_1, \dots, x_n \}$, supongamos sin perdida de
  generalidad que $x = x_m$. Entonces
  $\sigma x = \sigma t_k = \sigma \vec{S} x$ pues $\sigma \in U(S)$.

  Para el caso $x \not \in \{x_1, \dots, x_n \}$. Entonces
  $\sigma x = \sigma \vec{S} x$, pues $\vec{S} x = x$.
\end{demo}

\begin{lema}
  Si $S$ está en su forma resuelta entonces $\vec{S}$ es un UMG
  idempotente de $S$.
\end{lema}

\begin{demo}
  Usando el lema \ref{lema:3.2}, deducimos que ninguno de los $x_i$
  ocurren en $t_i$. Por la misma razón tenemos que
  $\vec{S} x_i = t_i = \vec{S} t_i$, es decir $\vec{S} \in U(S)$.
  Finalmente $\vec{S}$ es un UMG por el lema \ref{lema:3.3},
  $\vec{S} \lesssim \sigma$ para todo $\sigma \in U(S)$.
\end{demo}

Es decir, una vez que tengamos la forma resuelta de $S$, podemos extraer
un UMG idempotente.

Vamos a dar una serie de reglas de transformación para conseguirlo.
\\
$
\begin{array}{ll}
  $Borrar$       & $Elimina ecuaciones triviales$ \\ 
  $Descomponer$  & $Reemplaza ecuaciones entre términos por ecuaciones entre $ \\
                 & $sus subtérminos$ \\ 
  $Orientar$     & $Mueve variables hacia el lado izquierdo$ \\ 
  $Resuelve$     & $Afianza las soluciones, eliminando la variable resuelta en el$ \\
                 & $resto del problema$
\end{array} 
$
\\
\\
Que formalmente significan,
\\
\\
$
\begin{array}{llll}
  $Borrar$      
    & \{t =^? t  \} \sqcup S                             
    & \Longrightarrow & S \\ 
  $Descomponer$ 
    & \{f(\overline{t_n}) =^? f(\overline{u_n}) \} \sqcup S  
    & \Longrightarrow & \{t_1 =^? u_1, \dots, t_n =^? u_n\} \cup S\\ 
  $Orientar$    
    & \{t =^? x \} \sqcup S                              
    & \Longrightarrow & \{x=^? t \} \cup S$ si $t \not\in V \\ 
  $Resuelve$    
    & \{x =^? t \} \sqcup S                              
    & \Longrightarrow & \{x=^? t \} \cup \{x \mapsto t \}(S) \\
    &                                                    
    &                 & $si $ x \in \Var(S) - \Var(t) \\
\end{array} 
\\
$ Donde $\sqcup$ denota la unión disjunta.

%Ejemplo

\begin{defi} 
  La función $\Unify(S)$ devuelve un unificador de máxima generalidad si
  existe, y falla si no. Tiene la siguiente definición, \\ 
  $\begin{array}{lll}
    $\textbf{\Unify(S)}$      
    & = & $Mientras exista un $T$ tal que $S \Longrightarrow T, S:= T$.$ \\
    & & $Si $S$ es una forma resuelta entonces devuelve $\vec{S}$, si no, falla.$ \\
   \end{array} 
  $
\end{defi}

Este algoritmo es no determinista, pues siempre se puede aplicar más
de una regla de transformación. Por ello, el algoritmo escogerá una
regla arbitraria. La terminación de $\Unify$ dependerá de la
terminación de $\Longrightarrow$.

\begin{lema} 
  $\Unify$ termina para todos las entradas.
\end{lema}

\begin{lema}
  Si $S \Longrightarrow T$ entonces $U(S) = U(T)$
\end{lema}

\begin{lema} 
  Si \Unify(S) devuelve una sustitución $\sigma$, $\sigma$ es un UMG
  idempotente de $S$
\end{lema}

% La prueba de S es resoluble, necesita de los dos siguientes
% enunciados

\begin{lema}
  Una ecuación $f(\sigma(\overline{s_m})) =^?  g(\overline{t_n})$
  donde $f \not = g$, no tiene solución.
\end{lema}

\begin{lema}
  Una ecuación $x =^ t$, donde $x \in \Var(t)$ y $x \not = t$, no
  tiene solución.
\end{lema}

\begin{lema} 
  Si $S$ es resoluble, $\Unify(S)$ no falla.
\end{lema}

\section{Implementación de la unificación y la reescritura de términos
  en Haskell}

En esta sección vamos a implementar en el lenguaje de programación
funcional Haskell los algoritmos de este capítulo. Usaremos algunas
funciones del capítulo anterior, que se encuentran en la librería
\texttt{Terminos.hs}.

Empezamos implementando los errores. Estos son causados por
\texttt{UNIFICACION} (si el sistema no tiene solución) o
\texttt{REGLA}\index{\texttt{ERROR}}(si durante la reescritura no se
puede aplicar ninguna regla más).
\begin{codigo}
data ERROR = UNIFICACION
           | REGLA
           deriving (Eq, Show)
\end{codigo}

Una ecuación es un par de términos \index{\texttt{Ecuacion}}
\begin{codigo}
type Ecuacion = (Termino,Termino)
\end{codigo}

Un sistema de ecuaciones es una lista de ecuaciones \index{\texttt{Sistema}}
\begin{codigo}
type Sistema = [Ecuacion]
\end{codigo}

\subsection{Algoritmo de unificación}

\index{\texttt{unificacion}} \texttt{(unificacion t1 t2)} es
\begin{itemize}
\item \texttt{(Right s)} si los términos \texttt{t1} y \texttt{t2} son
  unificables y \texttt{s} es un unificador del máxima generalidad de
  \texttt{t1} y \texttt{t2}.
\item \texttt{(Left UNIFICACION)} si \texttt{t1} y \texttt{t2} no son
  unificables.
\end{itemize}
Por ejemplo,
\begin{sesion}
>>> unificacion (T "f" [V ("x",0), T "g" [V ("z",0)]])
                  (T "f" [T "g" [V ("y",0)], V ("x",0)])
Right [(("z",0),V ("y",0)),(("x",0),T "g" [V ("y",0)])]
>>> unificacion (T "f" [V ("x",0),T "b" []])
                  (T "f" [T "a" [],V ("y",0)])
Right [(("y",0),T "b" []),(("x",0),T "a" [])]
>>> unificacion (T "f" [V ("x",0),V ("x",0)])
                  (T "f" [T "a" [],T "b" []])
Left UNIFICACION
>>> unificacion (T "f" [V ("x",0),T "g" [V ("y",0)]])
                  (T "f" [V ("y",0),V ("x",0)])
Left UNIFICACION
\end{sesion}

Su código es,
\begin{codigo}
unificacion :: Termino -> Termino -> Either ERROR Sustitucion
unificacion t1 t2 = unificacionS [(t1,t2)] []
\end{codigo}
 


\index{\texttt{unificacionS}} \texttt{(unificacionS es s)} es
\begin{itemize}
\item \texttt{(Right s')} si el sistema \texttt{s(es)}, obtenido
  aplicando la sustitución \texttt{s} a la ecuaciones de \texttt{es},
  es unificable y \texttt{s'} es un unificador del máxima generalidad
  de \texttt{s(es)}.
\item \texttt{(Left UNIFICACION)} si \texttt{s(es)} no es unificable.
\end{itemize}
Por ejemplo,
\begin{sesion}
>>> unificacionS [(T "f" [V ("x",0), T "g" [V ("z",0)]],
                   T "f" [T "g" [V ("y",0)], V ("x",0)])]
                   []
Right [(("z",0),V ("y",0)),(("x",0),T "g" [V ("y",0)])]
>>> unificacionS [(T "f" [V ("x",0),T "b" []],
                   T "f" [T "a" [],V ("y",0)])]
                   []
Right [(("y",0),T "b" []),(("x",0),T "a" [])]
>>> unificacionS [(T "f" [V ("x",0),V ("x",0)],
                   T "f" [T "a" [],T "b" []])]
                   []
Left UNIFICACION
>>> unificacionS [(T "f" [V ("x",0),T "g" [V ("y",0)]],
                   T "f" [V ("y",0),V ("x",0)])]
                   []
Left UNIFICACION
\end{sesion}

Su código es,
\begin{codigo}
unificacionS :: Sistema -> Sustitucion -> Either ERROR Sustitucion
unificacionS [] s = Right s
unificacionS ((V x,t):ts) s 
  | V x == t  = unificacionS ts s
  | otherwise = reglaElimina x t ts s
unificacionS ((t,V x):ts) s = 
  reglaElimina x t ts s
unificacionS ((T f ts1, T g ts2):ts) s 
  | f == g    = unificacionS (zip ts1 ts2 ++ ts) s
  | otherwise = Left UNIFICACION
\end{codigo}

\index{\texttt{reglaElimina}} \texttt{(reglaElimina x t es s)} es
\begin{itemize}
\item \texttt{(Left UNIFICACION)}, si \texttt{x} ocurre en \texttt{t}.
\item \texttt{(Right s')} es la solución del sistema obtenido al
  aplicarle a \texttt{es} la sustitución [x/t] en el entorno obtenido
  componiendo la sustitución \texttt{s} con [x/t], en caso
  contrario.
\end{itemize}
Por ejemplo,
\begin{sesion}
>>> reglaElimina ("x",0) (T "f" [V ("x",0)]) [] []
Left UNIFICACION
>>> reglaElimina ("x",0) (T "f" [V ("x",1)]) [] []
Right [(("x",0),T "f" [V ("x",1)])]
>>> reglaElimina ("x",0) (T "f" [V ("x",1)]) [] [(("y",2),V ("x",0))]
Right [(("x",0),T "f" [V ("x",1)]),(("y",2),T "f" [V ("x",1)])]
>>> reglaElimina ("x",0) (T "f" [V ("x",1)]) [(V ("x",1),V ("x",0))] []
Left UNIFICACION
>>> reglaElimina ("x",0) (T "f" [V ("x",1)]) [(V ("x",1),V ("y",0))] []
Right [(("x",1),V ("y",0)),(("x",0),T "f" [V ("y",0)])]
>>> reglaElimina ("x",0) (T "f" [V ("x",1)]) [(V ("y",1),V ("x",0))] []
Right [(("y",1),T "f" [V ("x",1)]),(("x",0),T "f" [V ("x",1)])]
\end{sesion}

Su código es,
\begin{codigo}
reglaElimina :: Nvariable -> Termino -> Sistema -> Sustitucion 
             -> Either ERROR Sustitucion
reglaElimina x t es s 
  | ocurre x t = Left UNIFICACION
  | otherwise  = unificacionS es' s'
  where es' = [(aplicaTerm [(x,t)] t1, aplicaTerm [(x,t)] t2) 
                 | (t1,t2) <- es] 
        s'  = (x,t) : map (\(y,u) -> (y, aplicaTerm [(x,t)] u)) s
\end{codigo}
 
%¿Incluyo a modo de ejemplo lo obtenido con el Debug.Trace?

\subsection{Equiparación de términos}

\index{\texttt{equiparacion}} \texttt{(equiparacion t1 t2)} es
\begin{itemize}
\item \texttt{(Right s)} si \texttt{t1} y es una instancia de \texttt{t1} y \texttt{s} es un equiparador
  del máxima generalidad de \texttt{t1} con \texttt{t2}.
\item \texttt{(Left UNIFICACION)} en caso contrario.
\end{itemize}
Por ejemplo,
\begin{sesion}
>>> let t1 = T "f" [V ("x",0), T "a" []]
>>> let t2 = T "f" [T "b" [],  T "a" []]
>>> equiparacion t1 t2
Right [(("x",0),T "b" [])]
>>> equiparacion t2 t1
Left UNIFICACION
>>> let t3 = T "f" [V ("x",0), V ("y",0)]
>>> equiparacion t1 t3
Left UNIFICACION
\end{sesion}

Su código es,
\begin{codigo}
equiparacion :: Termino -> Termino -> Either ERROR Sustitucion
equiparacion t1 t2 = equiparacionS [(t1,t2)] []
\end{codigo}

\index{\texttt{equiparacionS}} \texttt{(equiparacionS es s)} es
\begin{itemize}
\item \texttt{(Right s')} si la derecha de las ecuaciones de
  \texttt{s(es)} y es una instancia de la izquierda y \texttt{s'} es
  un equiparador de máxima generalidad de la izquierda con la derecha
\item \texttt{(Left UNIFICACION)} en caso contrario.
\end{itemize}
Por ejemplo,
\begin{sesion}
>>> let t1 = T "f" [V ("x",0), T "a" []]
>>> let t2 = T "f" [T "b" [],  T "a" []]
>>> equiparacionS [(t1,t2)] []
Right [(("x",0),T "b" [])]
>>> equiparacionS [(t2,t1)] []
Left UNIFICACION
>>> let t3 = T "f" [V ("x",0), V ("y",0)]
>>> equiparacionS [(t1,t3)] []
Left UNIFICACION
\end{sesion}

Su código es,
\begin{codigo}
equiparacionS :: Sistema -> Sustitucion -> Either ERROR Sustitucion
equiparacionS [] s = Right s
equiparacionS ((V x,t):es) s 
    | not (enDominio x s)   = equiparacionS es ((x,t):s)
    | aplicaVar s x == t    = equiparacionS es s
    | otherwise             = Left UNIFICACION
equiparacionS ((_,V _):_) _ = Left UNIFICACION
equiparacionS ((T f ts1, T g ts2):es) s 
    | f == g    = equiparacionS (zip ts1 ts2 ++ es) s
    | otherwise = Left UNIFICACION
\end{codigo}

\subsection{Reescritura de términos}

\index{\texttt{reescribe}} \texttt{(reescribe es t)} es
\begin{itemize}
\item \texttt{(Right s')}, donde \texttt{s'} es el término obtenido
  reescribiendo \texttt{t} con la primera regla de es que con la que
  se pueda reescribir.
\item \texttt{(Left REGLA)} en caso contrario.
\end{itemize}
Por ejemplo,
\begin{sesion}
>>> reescribe [(V ("x",1), V ("y",1)),
                 (V ("x",2), V ("y",2))]
                 (V("x",2))
Right (V ("y",2))
>>> reescribe [(V ("x",1), V ("y",1)),
                 (V ("x",1), V ("y",2))]
                 (V("x",2))
Left REGLA
>>> reescribe [(V ("x",2),T "f" [V ("x",3)]),
                 (V ("x",3),T "f" [V ("x",4)])]
                 (V("x",2))
Right (T "f" [V ("x",3)])
>>> reescribe [(V ("x",1),T "a" [])]
                 (T "f" [V ("x",1), T "b" []])
Right (T "f" [T "a" [],T "b" []])
\end{sesion}

Su código es,
\begin{codigo}
reescribe :: Sistema -> Termino -> Either ERROR Termino
reescribe [] _ = Left REGLA
reescribe ((l,r):es) t
    | a == Left UNIFICACION = reescribe es t
    | b == t = reescribe es t
    | otherwise = Right b
    where a = equiparacion l r
          b = aplicaTerm (elimR a) t
          elimR (Right x) = x
\end{codigo}
 
\index{\texttt{formaNormal}} \texttt{(formaNormal es t)} es la forma normal de \texttt{t} respecto
de \texttt{es}. Por ejemplo,

\begin{sesion}
>>> formaNormal [(V ("x",1), V ("y",1)),
                   (V ("x",2), V ("y",2))]
                   (V("x",2))
V ("y",2)
>>> formaNormal [(V ("x",1), V ("y",1)),
                   (V ("x",2), V ("y",2))]
                   (V("x",3))
V ("x",3)
>>> formaNormal [(V ("z",1), V ("a",1)),
                   (V ("x",1), T "g" [V ("z",1)])]
                   (T "f" [V ("x",1)])
T "f" [T "g" [V ("a",1)]]
\end{sesion}

Su código es,
\begin{codigo}
formaNormal :: Sistema -> Termino -> Termino
formaNormal es (V x)
  | isRight a = elimR a
  | otherwise = V x
  where a = reescribe es (V x)
        elimR (Right r) = r
formaNormal es (T f ts)
  | a == Left REGLA = u
  | otherwise       = formaNormal es (elimR a)
  where u = T f (map (formaNormal es) ts)
        a = reescribe es u
        elimR (Right r) = r
\end{codigo}






 
%%% Local Variables:
%%% mode: latex
%%% TeX-master: "SRT_en_Haskell"
%%% End:
