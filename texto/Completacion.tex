\chapter{Completación}

Durante los anteriores capítulos hemos estudiado la decibilidad para
el problema de la terminación y confluencia. Hemos probado que ambos
problemas son indecidibles para el caso general. En este capítulo
vamos a estudiar un procedimiento, para que dado un sistema de
reescritura, modificar sus elementos para lograr que sea terminante y
confluente. Formalmente queremos construir un prodecimiento para el
problema de la palabra para un sistema finito de identidades $E$.

Supongamos que tenemos un conjunto
$E := \{ x + 0 \equiv x, x + s(y) \equiv s(x+y) \}$. El sistema de
reescritura generado por $E$ es
$R := \{ x + 0 \rightarrow x, x + s(y) \rightarrow s(x+y)\}$.  Para
probar la termianción usamos el camino de orden lexicográfico inducido
por $>$. La confluencia se verifica pues $R$ no tiene pares críticos.

Para estudiar este conjunto hemos aplicado condiciones suficientes, es
decir, podemos tener un sistema de identidades al que no podamos
aplicar las propiedades que hemos estudiado, y sin embargo sea
terminante y/o confluente. La idea detrás de los algorítmos es añadir
reglas que no cambien la estructura equacional de $E$.

Por ejemplo, supongamos que tenemos un sistema $E$ que es terminante
pero no confluente. Eso quiere decir que existen pares críticos en el
sistema de reescritura $R$ asociado a $E$. Si añadimos alguna regla
adiccional, podemos conseguir que esos pares críticos sean unibles.

Hay que tener en cuenta que al añadir reglas adiccionales se crearán
nuevos pares críticos. Este proceso continuará hasta que no existan
más pares críticos.

\section{Algoritmo de completación}

Vamos a definir un proceso que empieza por un conjunto finito de
identidades $E$, que intenta encontrar un sistema de reescritura
convergente $R$ que sea equivalente a $E$.

\textbf{Algoritmo} \hrulefill

\textbf{Input:} Un conjunto finito $E$ de $\Sigma$-identidades y un
orden de reducción > en $T(\Sigma, V)$.

\textbf{Output:} Si el procedimiento termina; un sistema de
reescritura $R$ finito y convergente que es equivalente a $E$. En caso
contrario devuelve un \texttt{Error}.

\textbf{Inicialización:}
Si existe una identidad $(s \equiv t) \in E$ tal que
$s \not = t, s \not > t$ y $t \not > s$, entonces termina con
\texttt{Error}.
En otro caso,
$i := 0$
$R_0 := \{l \rightarrow r | (l \equiv r) \in E \cup E^{-1} \wedge l > r \}$.

\texttt{Mientras} $R_i \not = R_{i-1}$

$R_{i+1} := R_i$;

Para todo $\langle s, t \rangle \in$ \texttt{ParesCriticos($R_i$)}:

$>$ Reducir $s,t$ a forma normal $s', t'$

$>$ Si $s' \not = t'$, $s' \not > t'$ y $t' \not > s'$ terminar con \texttt{Error}

$>$ Si $s' > t'$ entonces $R_{i+1} := R_{i+1} \cup \{s' \rightarrow t'\}$

$>$ Si $t' > s'$ entonces $R_{i+1} := R_{i+1} \cup \{t' \rightarrow s'\}$

$i := i+1$

Devuelve $R_i$

\hrulefill

Primero eliminamos todas las identidades $t = t$ y calculamos todos
los pares críticos y vamos recorriendo la lista. Cada término
$\langle s, t \rangle$ se reduce a su forma normal $s'$ y $t'$. Si
estos últimos son iguales, el par crítico es unible y se pasa al
siguiente. En caso contrario, transformamos el par crítico en una
regla de reescritura. Se comprueba la terminación de esa regla con
$>$. Si esta última se verifica, se añade al sistema de
reescritura. El proceso termina cuando el sistema no tenga más pares
críticos unibles, ó produzca un \texttt{Error}.

El algoritmo puede producir varias situaciones; que devuelva un
sistema de reescritura en la interacción donde todos los pares
críticos son unibles, que produzca un \texttt{Error}, o que no pare
pues se generen infinitas nuevas reglas.

Por último, vamos a demostrar que el resultado del algoritmo es una
solución para el problema de la palabra.

\begin{teor}
  Sea $E$ un conjunto finito de identidades, y $>$ un orden de
  reducción.

  \begin{enumerate}
  \item Si el resultado del algoritmo aplicado a $E$ y al orden $>$
    termina y devuelve $R_n$, entonces $R_n$ es un sistema de
    reescritura finito convergente equivalente a $E$. Además $R_n$ es
    un procedimiento para el problema de la parada de $E$.

  \item Si el resultado del algoritmo aplicado a $E$ y al orden $>$ no
    termina, entonces $R_\infty := \cap_{i \geq 0} R_i$ es un sistema
    de reescritura infinito convergente que es equivalente a $E$.
  \end{enumerate}

\end{teor}

\begin{demo}
  (1) $R_n$ es finito pues se construye mediante iteraciones finitas,
  donde cada interacción añade reglas $l \rightarrow r$ que satisfacen
  $l > r$. Por tanto por construcción $R_n$ es terminante. Además es
  confluente pues todos sus pares críticos son unibles.

  Por doble inclusión probaremos $\approx_{R_n} = \approx_E$.  La
  primera inclusión se da pues $\approx_{R_0} = \approx_E$ es cierto y
  $R_0 \subseteq R_1 \subseteq \dots \subseteq R_n$.

  La segunda se demuestra mediante inducción para
  $\approx_{R_i} \subseteq \approx_E$ en $i$.  El caso base es
  trivial. Para el paso de la inducción, usaremos que $s \approx_R t$
  es cierto para cada par crítico $\langle s, t \rangle$ de $R_i$, y
  como la forma normal $s'$, $t'$ de $s$ y $t$, verifica
  $s \approx_{R_i} s'$ y $t \approx_{R_i} t'$. Por tanto, tenemos
  $s' \approx_R t'$ que implica $s' \approx_E t'$ por la hipótesis de
  inducción, y que $ \approx_{R_i} \subseteq \approx_E$ es decidible
  por \ref{teor:3.1}.

  (2) $R_{\infty }$ es un sistema infinito donde en cada interacción
  se añade al menos una regla. La equivalencia de $R_{\infty }$ y $E$,
  y la terminación de $R_{\infty }$ se prueba de manera análoga a (1).

  Para probar la confluencia de $R_{\infty }$ vamos a probar que todos
  los pares críticos son unibles. Sea $\langle s,t \rangle$ un par
  crítico entre las reglas $l \rightarrow r$,
  $g \rightarrow d \in R_{\infty }$. Como
  $R_{\infty } = \cup_{i \geq 0} R_i$ y
  $R_0 \subseteq R_1 \subseteq \dots$, esto significa que existe
  $n \geq 0$ tal que $l \rightarrow r$, $g \rightarrow d \in R_n$, que
  implica que $\langle s,t \rangle \in$
  \texttt{paresCriticos($R_n$)}. Este par es unible en $R_n$ ó es
  unible en $R_{n+1}$, pues se le habrá añadido la correspondiente
  regla. Por tanto se da la convergencia.
\end{demo}

\section{Mejorando la completación}

El anterior algoritmo suele generar un gran número de reglas. Solo hay
que ver que para cada par crítico surgen varias. Además hay que hacer
el calculo de los pares críticos en cada paso, lo que hace que el algoritmo sea muy costoso tanto para el tiempo como para la memoria.

Para mejorar el algoritmo de completación, necesitamos varios
resultados que nos aseguren de que el proceso es correcto. En la
siguiente sección hablaremos del algoritmo de Huet. Este algoritmo es
algo mas complejo que el anterior y necesitaremos estos resultados
para comprobar que efectivamente, el proceso funciona.

La idea se basa en extender las reglas de simplificación por un
conjunto de reglas que nos resulten mas beneficiosas. Las reglas son
las siguientes.

\[
  \begin{array}{lcl}
    $\texttt{Deducir} $ & \dfrac{E,R}{E \cup \{s \approx t \}, R} & $Si $ s \leftarrow_R u \rightarrow_R t \\ \\
    $\texttt{Orientar}$ & \dfrac{E \cup \{s \approx' t\},R}{E,R \cup \{s \rightarrow t\}} & $Si $ s > t \\ \\
    $\texttt{Eliminar}$ & \dfrac{E \cup \{s \approx s\}, R}{E,R} & \\ \\
    $\texttt{Simplificar identidad}$   & \dfrac{E \cup \{ s\approx' t\},R}{E \cup \{ u \approx t\},R} & $Si $ s \rightarrow_R u\\ \\
    $\texttt{D-simplificar regla}$    & \dfrac{E, R \cup \{s \rightarrow t \} }{E,R \cup \{s \rightarrow u\}} & $Si $ t \rightarrow_R u   \\ \\
    $\texttt{I-simplificar regla}$    & \dfrac{E, R \cup \{s \rightarrow t \} }{E \cup \{u \approx t \}, R} & $Si $ s \xrightarrow{\sqsupset}_R u
  \end{array} 
\]

La notación $s \approx' t$ indica que es un par no ordenado, es decir
$s \approx t \in E$ ó $t \approx s \in R$. Denotaremos
$(E,R) \vdash_C (E',R')$ si $(E,R)$ se puede transformar en $(E', R')$
aplicando alguna regla anterior.

Las reglas generan un sistema terminante de reescritura terminante,
pues todas están orientadas con el orden de reducción $>$, de lo que
deducimos el siguiente resultado.

\begin{lema}
 Si $R \subseteq >$ y $(E,R) \vdash_C (E',R')$ entonces $R' \subseteq >$.
\end{lema}

Una consecuencia directa de este lema es dado un sistema de
reescritura $R$ del par $(E,R)$ es terminante si este par se ha
obtenido por uno inicial de la forma $(E_0, \emptyset)$ aplicando las reglas.

El siguiente resultado demuestra que las reglas no cambian la
estructura ecuacional.

\begin{lema}
  $(E_1,R_1) \vdash_C (E_2,R_2)$ implica
  $\approx_{E_1 \cup R_1} = \approx_{E_2 \cup R_2}$.
\end{lema}

\begin{demo}
  La demostración es trivial para las tres primeras reglas, por lo que
  nos centraremos en las tres últimas.
  \begin{itemize}
  \item Para \texttt{Simplificar identidad}, partiendo de que
    $E_1 = E \cup \{s \approx' t\}$, $E_2 = E \cup \{ u \approx t \}$,
    $R_1 = R = R_2$, y $s \rightarrow_R u$, que implica
    $u \approx_{E_1 \cup R_1} t$ y
    $\approx_{E_2 \cup R_2} \subseteq \approx_{E_1 \cup R_1}$. Para la
    otra inclusión partimos de $u \equiv t \in E_2$,
    $s \rightarrow_R u$ y $R = R_2$ implica
    $s \approx_{E_2 \cup R_2} t$ y por tanto,
    $\approx_{E_1 \cup R_1} \subseteq \approx_{E_2 \cup R_2}$.
  \item Para \texttt{D-simplificar regla}, tenemos $E_1 = E = E_2$,
    $R_1 = R \cup \{s \rightarrow t \}, R_2 = R \cup \{s \rightarrow
    u\}$ y $t \rightarrow_R u$. Con $s \rightarrow t \in R_1$,
    $t \rightarrow_R u$ y $R \subseteq R_1$ conseguimos
    $s \approx_{E_1 \cup R_1} u$, y con $s \rightarrow u \in R_2$,
    $t \rightarrow_R u$, y $R \subseteq R_2$ implica
    $s \approx_{E_2 \cup R_2} t$. Esto demuestra
    $\approx_{E_1 \cup R_1} = \approx_{E_2 \cup R_2}$.
  \item Para \texttt{I-simplificar regla} se demuestra de manera análoga
\end{itemize}
\end{demo}

Usando la idea del procedimiento de completación que hemos comentado
antes, formalizamos su definición.

\begin{defi}
  Un procedimiento de completación es un programa que acepta un
  conjunto finito de identidades $E_0$, un orden de reducción $>$, y
  usa las reglas para generar una secuencia (finita o infinita),

  \[
    (E_0,R_0) \vdash_C (E_1,R_1) \vdash_C (E_2,R_2) \vdash_C (E_3,
    R_3) \vdash_C \dots
  \]
  donde $R_0 := \emptyset$.  Además definimos $E_\omega$ como el
  conjunto de identidades persistentes y $R_\omega$ como el conjunto
  de reglas persistentes,

  \[
    E_\omega := \bigcup_{i \geq 0} \bigcap_{j \geq i} E_j
  \]

  \[
R_\omega :=  \bigcup_{i \geq 0} \bigcap_{j \geq i} R_j
  \]
  
\end{defi}

Diremos que la secuencia es un éxito syss $E_\omega = \emptyset$ y
$R_\omega$ es convergente y equivalente a $E_0$. Diremos que falla
syss $E_\omega \not = \emptyset$. Un procedimiento de completación es
correcto syss cada secuencia que no falle, sea un éxito.

Por último, la definición siguiente recoge la importancia de los pares
críticos a la hora de analizar los procedimientos de completación.

\begin{defi}
  Una secuencia de completación se dirá justa syss

  \[
\texttt{paresCriticos}(R_\omega) \subseteq \bigcup_{i\geq 0} E_i
\]
Un procedimiento de completación es justo syss cada secuencia de éxito
es justa.
\end{defi}

Y enunciamos el resultado que nos ayudará a probar que el algoritmo de
Huet resuelve el problema de completación.

\begin{teor}
  Cada procedimiento de completación justo es correcto.
\end{teor}


\section{Algoritmo de Huet}

El algoritmo que vamos a enunciar es una extensión no trivial del
algorítmo básico de completación.

\textbf{Algoritmo de Huet} \hrulefill

\textbf{Input:} Un conjunto finito $E$ de $\Sigma$-identidades y un
orden de reducción > en $T(\Sigma, V)$.

\textbf{Output:} Si el procedimiento termina; un sistema de
reescritura $R$ infinito limitado y convergente que es equivalente a
$E$. En caso contrario devuelve un \texttt{Error}.

\textbf{Inicialización:}
$R_0 := \emptyset, E_O := E, i:= 0;$

\texttt{Mientras} $E_i \not = \emptyset$ ó exista una regla no marcada
en $R_i$

\hspace{0.5cm} \texttt{Mientras} $E_i \not = \emptyset$

\hspace{0.5cm} (a) Escoge una identidad $s \approx t \in E_i$

\hspace{0.5cm} (b) Reduce $s,t$ a una forma $R_i$-normal $s',t'$

\hspace{0.5cm} (c) Si $s' = t'$ entonces,

\hspace{1.4cm} $R_{i+1} := R_i$

\hspace{1.4cm} $E_{i+1} := E_i - \{s \approx t\}$

\hspace{1.4cm} $i := i+1$

\hspace{0.5cm} (d) Si no, si $s' \not > t'$ y $t' \not s'$, entonces
terminar con un output \texttt{Error}.

\hspace{0.5cm} (e) Si no, sea $l,r$ tal que $\{l,r \} = \{ s',t' \}$ y
$l >r$;

\hspace{1.4cm}
$R_{i+1} := \{g \rightarrow d'$ $|$ $g \rightarrow d \in R_i,$ $g$ no
puede reducirse con $l \rightarrow r$, y $d'$ es una forma normal de
$d$ respecto $R_i \cup \{l \rightarrow r\} \}$

\hspace{1.4cm}
$E_{i+1} := (E_i - \{s \approx t \}) \cup \{g' \approx d$ $|$
$g \rightarrow d \in R_i$ y $g$ puede ser reducido a $g'$ con $l \rightarrow r \}$.

\hspace{1.4cm} i := i + 1

\hspace{0.5cm} Si existe una regla no marcada en $R_i$, sea
$l \rightarrow r$ esa regla.

\hspace{1.4cm} $R_{i+1} := R_i$

\hspace{1.4cm} $E_{i+1} := \{s \approx t$ $|$ $\langle s, t \rangle$
es un par crítico de $l \rightarrow r$ ó una regla marcada en $R_i \}$

\hspace{1.4cm} $i := i+1$

\hspace{1.4cm} Marca $l \rightarrow r$.

Devuelve $R_i$

\hrulefill

Comprobamos que el algoritmo efectivamente realiza un procedimiento de
completación sobre $E$

\begin{lema}
El algoritmo de completación de Huet es un procedimiento de completación.
\end{lema}

\begin{demo}
  El objetivo de la prueba es comprobar que cada paso del algoritmo es
  una de las reglas de la anterior sección.
  
  La computación de los pares críticos en el primer \texttt{Mientras}
  se puede realizar mediante \texttt{Deducir}, el paso (b) con
  \texttt{Simplificar identidad}, (c) con \texttt{Eliminar} y (e) con
  \texttt{Orienta}, \texttt{R-simplificar regla} y
  \texttt{L-simplificar regla}.
\end{demo}

Debemos comprobar que si el algoritmo devuelve un \texttt{Error}, el
procedimiento de completación falla

\begin{lema}
Una cadena del algoritmo de Huet falla syss termina con output \texttt{Fail}.
\end{lema}

\begin{demo}
  Si el procedimiento de Huet termina con \texttt{Error}, entonces hay
  una identidad que no puede transformarse por la regla de
  simplificación y orientación, entonces $E_\omega \not = \emptyset$.
  Si el procedimiento no termina con \texttt{Error}, entonces no hay
  identidades y por ello, el bucle interior siempre termina. Por ello
  vamos a asociar cada iteracción del bucle como un multiconjunto de
  pares,

  \[
    K_i := \{ ( \{s,t \},1) | s \approx t \in E_i \} \cup \{ ( \{l,r
    \},0) | l \rightarrow r \in R_i \}
  \]
  Donde la primera componente esta ordenada usando el orden de
  multiconjuntos inducido por el orden de reducción $>$, y la segunda
  componente por el producto del orden lexicográfico.

  Como los multiconjuntos $K_i$ están ordenados usando una extensión
  de este orden de pares, este esta bien fundando y $K_i >
  K_{i+1}$. Por tanto el bucle termina y se demuestra el resultado.
\end{demo}

\section{Implementación del algoritmo de Huet}

La implementación del algoritmo de Huet se realizará por partes.

\begin{itemize}

\item \texttt{(minRegla rl n r1 r2)} \index{\texttt{minRegla}} divide un
  conjunto de reglas en dos, donde uno tiene el tamaño de \texttt{r1}. Por
  ejemplo,
  \begin{sesion}
>>> minRegla (V("x",1),V("x",2)) 2 [(V("y",1),V("y",2))] []
((V ("x",1),V ("x",2)),[(V ("y",1),V ("y",2))])
>>> minRegla (V("x",1),V("x",2)) 1 [(V("y",1),V("y",2))] []
((V ("x",1),V ("x",2)),[(V ("y",1),V ("y",2))])
>>> minRegla (V ("x",1),V ("x",2)) 2
             [(V ("y",1),V ("y",2)),
              (T "f" [V("z",1),V("z",2)], V ("x",2))] []
((V ("x",1),V ("x",2)),
 [(T "f" [V ("z",1),V ("z",2)],V ("x",2)),(V ("y",1),V ("y",2))])
>>> minRegla (V ("x",1),V ("x",2)) 4 [(V ("y",1),V ("y",2)),
             (T "f" [V("z",1),V("z",2)], V ("x",2))] []
((V ("y",1),V ("y",2)),
 [(T "f" [V ("z",1),V ("z",2)],V ("x",2)),(V ("x",1),V ("x",2))])
  \end{sesion}

  Su código es,

  \begin{codigo}
minRegla :: (Termino, Termino) -> Int -> [(Termino, Termino)]
     -> [(Termino, Termino)]
     -> ((Termino, Termino), [(Termino, Termino)])
minRegla rl _ [] r2 = (rl,r2)
minRegla rl n ((l,r):r1) r2
    | m < n = minRegla (l,r) m r1 (rl:r2)
    | otherwise = minRegla rl n r1 ((l,r):r2)
    where m = longitudTerm l + longitudTerm r
  \end{codigo}

\item \texttt{(escogeRegla r)} \index{\texttt{escogeRegla}} calcula la regla
  que hay que aplicar durante el algoritmo de Huet. Por ejemplo,

  \begin{sesion}
>>> escogeRegla [(V("y",1),V("y",2))]
((V ("y",1),V ("y",2)),[])
>>> escogeRegla [(V ("y",1),V ("y",2)),
                 (T "f" [V("z",1),V("z",2)], V ("x",2))]
((V ("y",1),V ("y",2)),[(T "f" [V ("z",1),V ("z",2)],V ("x",2))])
  \end{sesion}

  Su código es,

  \begin{codigo}
escogeRegla :: [(Termino, Termino)]
            -> ((Termino, Termino), [(Termino, Termino)])
escogeRegla [] = error("No se puede aplicar ninguna regla")
escogeRegla ((l,r):r1) = minRegla (l,r)
                         (longitudTerm l + longitudTerm r)
                         r1 []
  \end{codigo}

\item \texttt{(anadeRegla (l,r) e s r)} \index{\texttt{anadeRegla}} es el paso
  (e) del algoritmo de Huet.

  \begin{codigo}
anadeRegla :: (Termino, Termino) -> [(Termino, Termino)]
              -> [Ecuacion] -> [Ecuacion]
              -> ([(Termino, Termino)], [(Termino, Termino)],
                 [(Termino, Termino)])
anadeRegla (l,r) e s r1 = (e2, (l,r):s1, r2)
    where (e1,s1) = simpl l r s r1 s e []
          (e2,r2) = simpl l r s r1 r1 e1 []
simpl
  :: Termino
     -> Termino
     -> [Ecuacion]
     -> [Ecuacion]
     -> [(Termino, Termino)]
     -> [(Termino, Termino)]
     -> [(Termino, Termino)]
     -> ([(Termino, Termino)], [(Termino, Termino)])
simpl _ _ _ _ [] a b = (a,b)
simpl l r s r1 ((g,d):u) e1 u1 
    | g1 == g =
        simpl l r s r1 u e1 ((g, d1):u1)
    | otherwise = simpl l r r1 s u ((g1,d):e1) u1
       where g1 = formaNormal [(l,r)] g
             d1 = formaNormal ((l,r):(r1++s)) d
\end{codigo}

\item \texttt{(orienta ord trip)} \index{\texttt{orienta}} es el segundo bucle
  del algoritmo de Huet.

  \begin{codigo}
orienta :: (Termino -> Termino -> Ordering)
           -> ([(Termino, Termino)],  [(Termino, Termino)],
               [(Termino, Termino)])
           -> ([(Termino, Termino)], [(Termino, Termino)])
orienta _ ([],ss,rr) = (ss,rr)
orienta ord ((s,t):ee,ss,rr)
    | s1 == t1 = orienta ord (ee,ss,rr)
    | ord s1 t1 == GT = orienta ord (anadeRegla (s1,t1)
                                              ee ss rr)
    | ord t1 s1 == GT = orienta ord (anadeRegla (t1,s1)
                                              ee ss rr)
    | otherwise = error("Error con la funcion ord")
    where s1 = formaNormal (rr++ss) s
          t1 = formaNormal (rr++ss) t
  \end{codigo}

\item \texttt{(completa ord ee)} \index{\texttt{completa}} es el resultado de aplicar a ee, el algoritmo de completación de Huet. Por ejemplo,

  \begin{sesion}
>>> let ord = ordenCamLex (ordenPorLista ["a","b","c","x","y","z"])
>>> completa ord [(V("x",1),V("x",1))]
[]
>>> completa ord [(V ("y",1),V ("y",2)),
                  (T "f" [V("y",1),V("y",2)], V ("x",2))]
*** Exception: Error con la funcion ord
>>> completa ord [(T "f" [V("x",1),V("x",2)], V ("x",2))]
[(T "f" [V ("x",1),V ("x",2)],V ("x",2))]
\end{sesion}

Su código es,

\begin{codigo}
completa :: (Termino -> Termino -> Ordering)
            -> [(Termino, Termino)] -> [(Termino, Termino)]
completa ord ee = compl ord (ee,[],[])
compl :: (Termino -> Termino -> Ordering)
         -> ([(Termino, Termino)], [(Termino, Termino)],
             [(Termino, Termino)])
         -> [(Termino, Termino)]
compl ord (ee,ss,tt) = case orienta ord (ee,ss,tt) of
                         ([],rr1)  -> rr1
                         (ss1,rr1) -> compl ord
                                      (paCr,ss2,rl:rr1)
      where (rl, ss2) = escogeRegla ss1
            paCr = (paresLista2 [rl] rr1)++
                   (paresLista2 rr1 [rl])++
                   (paresLista2 [rl] [rl])
\end{codigo}

\end{itemize}

           


%%% Local Variables:
%%% mode: latex
%%% TeX-master: "SRT_en_Haskell"
%%% End:
