\chapter{Terminación}

Como hemos podido comprobar en los anteriores capítulos, es importante
que nuestros sistemas de reescritura tengan la propiedad de la
terminación. Sin embargo, en la primera sección de este capítulo
demostraremos que el problema de decidir si un sistema de reescritura
es terminante, es indecidible. Aunque en casos mas restringidos si
podremos decidir sobre la terminación del sistema de reescritura. En
la segunda sección definiremos los órdenes de reducción, obteniendo
una propiedad con ellos para verificar la terminación. En el resto
daremos diferentes maneras de definir los órdenes de reducción.

\section{El problema de decisión}

\subsection{La indecibilidad para el caso general}

En esta sección vamos a introducir conceptos generales de las Ciencias
de la Computación, aplicados a los sistemas de reescritura. Uno de los
problemas generales de las Ciencias de la Computación es afirmar si un
problema es decidible. Usaremos algunos de sus resultados sobre
máquinas de Turing. Supondremos, sin perdida de generalidad, que el
modelo es no determinista de una banda finita en ambas direcciones.

\begin{defi} 
  Una máquina de Turing no determinista $M$ viene dada por,
  \begin{enumerate}
  \item Un alfabeto $\Gamma := \{s_0, \dots, s_n \}$ de símbolos,
    donde $s_0$ es considerado el espacio en blanco.
  \item Un conjunto finito $Q = \{q_0, \dots, q_p \}$ de estados.
  \item Una relación de transición
    $\Delta \subseteq Q \times \Gamma \times Q \times \Gamma \times \{
    l,r \} $.
  \end{enumerate}
\end{defi}
% Dibujo de una maquina de turing 

Una maquina de Turing se puede interpretar de la siguiente
manera. Imaginemos que tenemos una tira de papel dividida en infinitos
cuadrados en ambas direcciones, y en cada cuadrado se encuentra un símbolo del
alfabeto. Además de la tira de papel, se encuentra un marcador en uno
de los cuadrados. A partir de la relación de transición que escojamos,
el marcador se moverá de un lado a otro y cambiará los símbolos del
papel. 

Un ejemplo de relación de transición es $\{ q_1, s_1, q_2, s_2, l
\}$. El marcador empieza por el cuadrado inicial de la tira de
papel. Si el marcador posee el estado $q_1$ y en ese cuadrado de
papel se encuentra $s_1$, procedemos a aplicar el algoritmo. Cambiamos
el estado del marcador por $q_2$ y el símbolo del trozo de papel por
$s_1$. A continuación movemos el marcador a la izquierda (si fuera
$l$) o derecha (si fuera $r$).

%EJEMPLO

Llamaremos a un estado de la computación de la máquina,
configuración. En la configuración de una máquina se incluye, la tira
de papel, la posición y el estado del marcador. Podemos expresar
$K \vdash_M K'$ si la configuración de $K'$ se puede obtener mediante
$K$ con una computación de la máquina de Turing $M$.

De aquí surge el problema de la parada. Nos interesa saber si dado una
maquina de Turing $M$ y una configuración de la máquina $K$,
termina. Es decir, que no ocurre $K \vdash_M K_1 \vdash_M K_2 \dots$.

Volviendo al tema de la reescritura, en nuestro es caso el problema es
un poco más difícil. Como podemos aplicar toda regla en cada momento,
no tenemos una configuración $K$ como en el problema de la
parada. Pero si planteamos el problema suponiendo que no partimos de
una configuración inicial $K$, la pregunta pasa a ser; ¿qué
configuraciones $K$ hace que el problema termine o no? A esto se le
denota por el problema de la parada uniforme. 

Para extrapolar lo que ya sabemos sobre máquinas de Turing a la
reescritura, vamos a codificar las configuraciones como términos con
una signatura $\Sigma_M$.

\begin{defi}
Sea $M$ una máquina de Turing, definimos 
\[
  \Sigma_M := \{ \overrightarrow{s_0}, \dots, \overrightarrow{s_n},
  \overleftarrow{s_0}, \dots, \overleftarrow{s_n} \} \cup \{q_0,
  \dots, q_p \} \cup \{\overrightarrow{l}, r \}
\]
donde cada función es de aridad 1.
\end{defi}

A continuación, enunciamos una definición que relaciona términos con
máquinas de Turing.

\begin{defi}
  Sea $x_0$ una variable fija. Un término de configuración en
  $\Sigma_M$ es cualquier término de la forma,
\[
  \overrightarrow{l}(\overrightarrow{s_{i_k}}(\dots
  \overrightarrow{s_{i_1}}(q(\overleftarrow{s_{j_1}}(\dots
  \overrightarrow{s_{j_h}}(\overleftarrow{r}(x_0))\dots)))\dots))
\]
donde $k,h \geq 0$, $\{i_1, \dots, i_k, j_1, \dots, j_h \} \subseteq \{0, \dots, n \}$, y $q \in Q$.
\end{defi}

Para entender mejor esta definición, debemos hacer las siguientes
consideraciones. El marcador se encuentra en $s_{j_1}$ con el estado
$q$. Los elementos que se encuentran a la derecha del marcador son
$s_{j_2}, \dots s_{j_h}$, y los elementos que se encuentran a su
derecha, son $s_{i_1}, \dots, s_{i_k}$. 

Como la tira es infinita, $\overrightarrow{l}, \overleftarrow{r}$ nos
ayudan a controlar los espacios vacíos. Si durante la computación de
la máquina de Turing hiciese falta un espacio en blanco, estas
funciones se lo proporcionan.

%Insertar dibujo con la transformación

Acabamos de transformar una máquina de Turing en un término. En la
siguiente definición, adaptaremos un sistema de reescritura para una
máquina de Turing.

\begin{defi}
  Un sistema de reescritura $R_M$ consiste en las siguientes reglas de
  reescritura,
\begin{itemize}
\item Para cada transición $(q, s_i, q', s_j, r) \in \Delta$, $R_M$
  contiene la regla,
\[
  q(\overleftarrow{s_i}(x)) \rightarrow \overrightarrow{s_j}(q'(x))
\]
Si $i=0$, entonces $R_M$ contiene la siguiente regla,
\[
  q(\overleftarrow{r}(x)) \rightarrow
  \overrightarrow{s_j}(q'(\overleftarrow{r}(x)))
\]
\item Para cada transición $(q,s_i,q',s_j,l) \in \Delta$, $R_M$
  contiene la regla,
\[
  \overrightarrow{l}(q(\overleftarrow{s_i}(x))) \rightarrow
  \overrightarrow{l}(q'(\overleftarrow{s_0}(\overleftarrow{s_j}(x))))
\]
y para cada $s_k \in \Gamma$ la regla,
\[
  \overrightarrow{s_k}(q(\overleftarrow{s_i}(x))) \rightarrow
  q'(\overleftarrow{s_k}(\overleftarrow{s_j}(x)))
\]
Si $i=0$, entonces $R_M$ contiene una regla adicional,
\[
  \overrightarrow{l}(q(\overleftarrow{r}(x))) \rightarrow
  \overrightarrow{l}(q'(\overleftarrow{s_0}(\overleftarrow{s_j}(\overleftarrow{r}(x)))))
\]
y para cada $s_k \in \Gamma$ la regla
\[
  \overrightarrow{s_k}(q(\overleftarrow{r}(x))) \rightarrow
  q'(\overleftarrow{s_0}(\overleftarrow{s_j}(\overleftarrow{r}(x)))))
\]
\end{itemize}
\end{defi}

Con estas reglas podemos llegar a la conclusión de que para todo par
de términos de configuración $t, t'$, que verifiquen
$t \rightarrow_{R_M} t'$, implica $K_t \vdash K_{t'}$. E igual ocurre
en sentido contrario, si tenemos dos configuraciones $K, K'$ y un
término de configuración $t$, si $K \vdash K'$ y $K \equiv K_t$, esto
implica que existe un término de configuración $t'$ tal que
$K' \equiv K_{t'}$ y $t \rightarrow_{R_M} t'$. \label{prop:4}

De aquí podemos razonar una propiedad. Como el problema de la parada
de las máquinas de Turing es indecidible, y acabamos de probar que una
máquina de Turing es equivalente a un sistema de reescritura (con las
reglas que hemos pedido), aseguramos que dado un sistema de
reescritura finito $R$ y un término $t$, el problema de ver si todas
las reducciones son terminantes empezando por $t$ es indecidible.

Sin embargo, el problema que acabamos de resolver no es el original
que queríamos. El problema de la terminación pide que todas las
reducciones desde todos los posibles términos sean terminantes. Puede
darse el caso que tengamos una configuración de términos que termine,
pero una reducción no terminante que empiece por un término que no
este en la configuración inicial. Por tanto el problema no esta
resuelto todavía. El siguiente lema asegura que este caso no puede
ocurrir.

\begin{lema} \label{lema:4.1}
  Sea  $t$  un  término  de   $\Sigma_M$.   Si  existe  una  reducción
  $t \rightarrow_{R_M}  t_1 \rightarrow_{R_M} t_2  \rightarrow \dots$,
  entonces existe  un término  de configuración  $t'$ y  una reducción
  infinita $R_M$ empezando por $t'$.
\end{lema}

\begin{demo}
  Vamos a considerar
  $\Sigma_M = \overrightarrow{\Gamma} \cup \overleftarrow{\Gamma} \cup
  Q \{ \overleftarrow{r} , \overrightarrow{l} \} $, donde
  $\overrightarrow{\Gamma} := \{\overrightarrow{s_1}, \dots,
  \overrightarrow{s_n} \}$ y
  $\overleftarrow{\Gamma} := \{ \overleftarrow{s_1}, \dots,
  \overleftarrow{s_n} \}$. Cualquier elemento $w$ de $\Sigma_M$ puede
  ser escrito como composición de funciones tal que,
  $w = u_1 v_1 u_2 v_2 \dots u_q v_q u_{q+1}$, ya que $u_i, v_j$ son
  funciones de aridad 1.

  Como todas las reglas de reescritura de $R_M$ contienen un símbolo
  de $Q$, cualquier reducción aplicada a $w$, se hace dentro de las
  funciones $v$.

  Es decir, que si tomamos $w(x) \rightarrow_{R_M} w'(x)$, entonces
  existe un índice $j$, $1 \leq j \leq q$, y una palabra
  $v'_j \in \overrightarrow{\Gamma}^{*} Q \overleftarrow{\Gamma}^{*}$
  tal que
  $w'= u_1 v_1 u_2 v_2 \dots u_j v'_j u_{j+1} \dots u_q v_1 u_{q+1}$,
  y que
  $\overrightarrow{l} v_j \overleftarrow{r} (x_0) \rightarrow_{R_M}
  \overrightarrow{l} v'_j \overleftarrow{r}$
  
  Como $q$ es finito, esto implica que la reducción infinita que
  empieza por $w(x)$ produce una reducción infinita empezando por
  $\overrightarrow{l} v_j \overleftarrow{r} (x_0)$. Como
  $\overrightarrow{l} v_j \overleftarrow{r} (x_0)$ es un término de
  configuración, queda demostrado el lema.
\end{demo}

Finalmente, por el lema \ref{lema:4.1}, obtenemos el objetivo principal
de esta sección, la indecibilidad para el problema de terminación de
los sistemas de reescritura.


\subsection{Sistemas de reescritura básicos hacia la derecha}

En esta subsección vamos a analizar un SRT particular $R$, donde sus
reglas hacia la derecha son términos básicos. A estos SRT les
llamaremos básicos hacia la derecha, y probaremos que son terminantes.

\begin{lema}
  Sea $R$ un SRT finito básico hacia la derecha. Entonces son
  equivalentes,
  \begin{enumerate}
  \item $R$ no termina.
  \item Existe una regla $l \rightarrow r \in R$ y un término $t$ de
    manera que $r \xrightarrow{+}_R t$ y $t$ contiene a $r$ como
    subtérmino.
  \end{enumerate}
\end{lema}

\begin{demo}
  $(2 \Rightarrow 1)$ es cierto pues obtenemos una reducción finita;
  $r\xrightarrow{+}_R t = t[r]_p \xrightarrow{+}_R t[t]_p =
  t[t[r]_p]_p \xrightarrow{+}_R \dots$, donde $p$ es la posición $t|_p = r$.
  
  $(1 \Rightarrow 2)$ se demuestra mediante inducción por el cardinal
  de $R$.

  Suponemos que $|R| >0>$ y consideramos una reducción infinita
  $t_1 \rightarrow t_2 \rightarrow t_3 \rightarrow \dots$. Tenemos que
  probar que esta cadena no termina. Sin perdida de generalidad al
  menos una reducción ocurre en la posición $\epsilon$. Esto significa
  que existe un índice $i$, una regla $l \rightarrow r \in R$, y una
  sustitución $\sigma$, tal que $t_i = \sigma(l)$ y
  $t_{i+1} = \sigma(r) = r$. Luego hay una reducción infinita
  $r \rightarrow_R t_{i+2} \rightarrow_R t_{i+3} \rightarrow_R \dots$
  que empieza por $r$.

  Si la regla $l \rightarrow r$ no es usada en la reducción, entonces
  aplicamos la inducción a $R - \{l \rightarrow r\}$. En el caso en
que sí sea usada, eso significa que existe $j$ tal que $r$ ocurre en
$t_{i+j}$ y por tanto, obtenemos la segunda proposición.
\end{demo}

Luego si $R$ termina entonces $\rightarrow_R$ es globalmente finita,
ya que es una ramificación finita por el corolario \ref{globfin} y el
lema \ref{ramfin}. Por tanto todas las reducciones terminarían y, al
ser finitas, lo hacen en un numero finito de pasos. Con esta idea,
obtenemos el resultado clave de esta subsección.

\begin{teor}
  Para un SRT finito básico hacia la derecha, el problema de la
  terminación es decidible.
\end{teor}

\section{Órdenes de reducción}

En esta sección relacionaremos la definición de orden bien fundado con
el problema de la terminación.

En la sección previa, hemos probado que el problema de la terminación
es indecidible. Sin embargo podemos dar varios métodos para resolver
el problema. Estos métodos no son totalmente automatizados.

La idea básica para probar la terminación, se hace mediante un orden
bien fundado. Supongamos que $R$ es un sistema de reescritura finito,
$>$ es un orden estricto bien fundado en $T(\Sigma, X)$. Relacionando
los sistemas de reescritura con los órdenes, si $R$ es terminante,
diremos que $s \rightarrow_R t$ implica $s > t$. En vez de decidir $s > t$,
tan solo tendremos que comprobar las reglas de $R$. Para considerar
este nuevo acercamiento al problema necesitamos pedir ciertas
propiedades a $>$.

\begin{defi}
  Sea $\Sigma$ una signatura y $V$ un conjunto numerable de
  variables. Un orden estricto $>$ es un orden de reescritura syss
  \begin{enumerate}
  \item Es compatible con las $Sigma$-operaciones: para todo
    $s_1,s_2 \in T(\Sigma, V)$, todo $n \geq 0$ y
    $f \in \Sigma^{(n)}$, $s_1 > s_2$ implica
    \[ f(t_1, \dots, t_{n-1}, s_1, t_{i+1}, \dots, t_n) > f(t_1,
      \dots, t_{n-1}, s_2, t_{i+1}, \dots, t_n) \]
    $\forall i, 1 \leq i \leq n$ y
    $\forall t_1, \dots, t_{i-1}, t_{i+1}, \dots, t_n \in T(\Sigma,
    V)$. %Hay una condición equivalente
  \item Es cerrado bajo sustituciones:
    $\forall \sigma \in \Sub(T(\Sigma,V))$, si $s_1 > s_2$ entonces
    $\sigma(s_1) > \sigma(s_2)$.
  \end{enumerate}
  Un orden de reducción es un orden de reescritura bien fundado.
\end{defi}

%Ejemplo

El siguiente teorema es de especial importancia para los órdenes de
reducción.

\begin{teor} 
  Un sistema de reescritura $R$ termina syss existe un orden de
  reducción $>$ que satisface $l>r$ para toda regla
  $l \rightarrow r \in R$.
\end{teor}

\begin{demo}
  $(\Rightarrow)$ Suponiendo que $R$ termina, entonces
  $\xrightarrow{+}_R$ es un orden de reducción ya que satisface
  $l \xrightarrow{x}_R r$ para todo $l \rightarrow r \in R$.

  $(\Leftarrow)$ Como $>$ es un orden de reescritura, para $l > r$ se
  verifica $t [ \sigma(l) ]_p > t [ \sigma(r) ]_p$ para todo
  término $t$, sustitución $\sigma$ y posición $p$. Como $>$ esta bien
  fundada, y para toda regla de $R$, $s_1 \rightarrow_R s_2$ implica
  $s_1 > s_2$, entonces no puede ocurrir una reducción infinita
  $s_1 \rightarrow_R s_2 \rightarrow_R s_3 \dots$.
\end{demo}

%\begin{defi}
%  Un orden de reducción $>$ en $T(\Sigma, V)$ es total en términos
%  básicos, si restringido a $T(\Sigma, \emptyset)$ es un orden linear
%  estricto.
%\end{defi}

A partir de aquí podemos dar varios métodos para dar una repuesta al
problema. En las secciones posteriores nos centraremos en los ordenes
de simplificación, orden de camino lexicográfico y orden de camino
recursivo, y daremos una implementación de ambos.

\section{Órdenes de simplificación}

En esta sección definiremos que son los órdenes de simplificación y
daremos dos órdenes construidos a partir de estos a modo de ejemplo en
las siguientes subsecciones. Empezamos por la definición,

\begin{defi}
  Sea $>$ un orden estricto en $T(\Sigma, V)$ es un orden de
  simplificación, syss es un orden de reescritura que para todo
  término $t \in T(\Sigma, V)$ y todas las posiciones
  $p \in \Pos(t) - \{ \epsilon \}$ entonces $t > t|_p$.
\end{defi}

La definición es equivalente a pedir que, para todo $n \leq 1$, todos
los símbolos de funciones $f \in \Sigma^{(n)}$, todas las variables
$x_1, \dots, x_n \in V$, tenemos que
$f(x_1, \dots, x_i, \dots, x_n) > x_i$. Usando esta nueva definición e
introduciendo algunos conceptos de álgebra de homomormismos, se puede
demostrar que los órdenes de simplificación son equivalentes a los
ordenes de reducción para $\Sigma$ finito.

\subsection{Órdenes de caminos lexicográficos}

La idea de los caminos recursivos reside en comparar las raíces de los
términos, y en comparar recursivamente sus subtérminos. Estos
subtérminos se pueden comparar mediante multiconjuntos (orden de
caminos de multiconjuntos), tuplas (orden de caminos lexicográficos),
o una mezcla de ambos (orden de caminos recursivos). Nos interesaremos
por estos dos últimos.

\begin{defi}
  Sea $\Sigma$ una signatura finita y $>$ un orden estricto de
  $\Sigma$. El orden de caminos lexicográficos $>_{ocl}$, se define
  como, $s>_{ocl}t$ syss ocurre alguno de los siguientes casos,
  \begin{itemize}
  \item (OCL1) $t \in \Var(s)$ y $s \not = t$ ó
  \item (OCL2) $s = f(s_1, \dots, s_m)$, $t = g(t_1, \dots, t_n)$, y
    \begin{itemize}
    \item (OCL2a) existe $i$, $1 \leq i \leq m$, con
      $s_i \geq_{lpo} t$ ó
    \item (OCL2b) $f > g$ y $s >_{ocl} t_j$ para todo $j$ en
      $1 \leq j \leq n$ ó
    \item (OCL2c) $f = g$, $s >_{lpo} t_j$ para todo $j$ en
      $1 \leq j \leq n$, y exista $i$, $1 \leq i \leq m$ tal que
      $s_1 = t_1, \dots, s_{i-1} = t_{i-1}$ y $s_i >_{ocl} t_i$
    \end{itemize}
  \end{itemize}
\end{defi}

La definición es recursiva y esta bien definida.

% INTRODUCIR EJEMPLO

\begin{teor}
  Para cualquier orden estricto $>$ en $\Sigma$, el orden de camino
  lexicográfico inducido es un orden de simplificación en
  $T(\Sigma, V)$.
\end{teor}

% FALTA DEMOSTRACIÓN

A continuación implementaremos el orden de caminos lexicográfico en
Haskell. Para realizar los ejemplos de esta función, primero
implementaremos \texttt{ordenPorLista xs a b}, que es el resultado de
comparar $a$ y $b$, tal que $a > b$ syss $a$ aparece antes en $x$s que
$b$. Por ejemplo,

\begin{sesion}
ghci> ordenPorLista ["a","b","c"] "a" "c"
GT
ghci> ordenPorLista ["a","b","c"] "c" "b"
LT
ghci> ordenPorLista ["a","b","c"] "b" "b"
EQ
\end{sesion}

Su código es,

\begin{codigo}
ordenPorLista :: Ord a => [a] -> a  -> a -> Ordering
ordenPorLista [] _ _ =
  error("Ninguno de los dos elementos se encuentran
         en la lista")
ordenPorLista (x:xs) a b
  | a == x = if a == b
             then EQ
             else GT
  | b == x = LT
  | otherwise = ordenPorLista xs a b
\end{codigo}

Definiremos \texttt{ordCamLex ord s t}, que es el resultado de
comparar $s$ y $t$ términos, con el orden de caminos lexicográfico
inducido por \texttt{ord}. Por ejemplo

\begin{sesion}
ghci> let ord = ordenPorLista ["i","f","e"]
ghci> ordenCamLex ord (T "f" [V ("x",1), T "e" []]) (V ("x",1))
GT
ghci> ordenCamLex ord (T "i" [T "e" []]) (T "e" [])
GT
ghci> ordenCamLex ord (T "i" [T "f" [V("x",1),V("y",1)]])
                      (T "f" [T "i" [V("y",1)], T "i" [V("x",1)]])
GT
ghci> ordenCamLex ord (T "f" [V("y",1),V("z",1)])
                      (T "f" [T "f" [V("x",1),V("y",1)], V("z",1)])
LT
\end{sesion}

Su código es,

\begin{codigo}
 ordenCamLex:: ([Char] -> [Char] -> Ordering)
                -> Termino -> Termino -> Ordering
ordenCamLex _ s (V x)
  | s == (V x) = EQ
  | ocurre x s = GT --OCL1
  | otherwise = LT
ordenCamLex _ (V _) (T _ _) = LT
ordenCamLex ord s@(T f ss) t@(T g ts) --OCL2
  | all (\x -> ordenCamLex ord x t == LT) ss
    = case ord f g of
      GT -> if all (\x -> ordenCamLex ord s x == GT) ts
            then GT --OCL2b
            else LT
      EQ -> if all (\x -> ordenCamLex ord s x == GT) ts
            then ordLex (ordenCamLex ord) ss ts --OCL2c
            else LT
      LT -> LT  
  | otherwise = GT --OCL1a
\end{codigo}

\subsection{Órdenes de caminos recursivos}


%%%%%%%%%%%%%%%%%%%%%%%%%%%%%%%%%%%%%%%%%%%%%%%%%%%
%%Apéndices
%%%%%%%%%%%%%%%%%%%%%%%%%%%%%%%%%%%%%%%%%%%%%%%%%%%

\clearpage
\addappheadtotoc
\appendix


%%% Local Variables:
%%% mode: latex
%%% TeX-master: "SRT_en_Haskell"
%%% End:
