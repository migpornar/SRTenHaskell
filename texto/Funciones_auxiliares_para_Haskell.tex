\chapter{Funciones auxiliares para Haskell}

Usaremos las siguientes funciones de la libreria Data.List

\begin{itemize}
\item \texttt{(xs ++ ys)} es la concatenación de \texttt{xs} e \texttt{ys}. Por ejemplo,
        
\begin{sesion}
> [2,5] ++ [3,7,6]
[2,5,3,7,6]
\end{sesion}
        
\item \texttt{(any p xs)} se verifica si algún elemento de \texttt{xs} cumple
  la propiedad \texttt{p}. Por ejemplo,

\begin{sesion}
> any even [3,2,5]
True
> any even [3,1,5]
False
\end{sesion}      

\item \texttt{(all p xs)} se verifica si todos los elementos de \texttt{xs}
  cumplen la propiedad \texttt{p}. Por ejemplo,

\begin{sesion}
> all even [2,6,8]
True
> all even [2,5,8]
False
\end{sesion}      

\item \texttt{(null xs)} se verifica si \texttt{xs} es la lista vacía. Por
  ejemplo, 

\begin{sesion}    
> null []
True
> null [3]
False
\end{sesion}      

\item \texttt{(zip xs ys)} es la lista de pares formado por los
  correspondientes elementos de \texttt{xs} e \texttt{ys}.

\begin{sesion}    
> zip [3,5,2] [4,7]
[(3,4),(5,7)]
> zip [3,5] [4,7,2]
[(3,4),(5,7)]
\end{sesion}              
\end{itemize}

%%% Local Variables:
%%% mode: latex
%%% TeX-master: "SRT_en_Haskell"
%%% End:
