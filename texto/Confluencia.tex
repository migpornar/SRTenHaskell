\chapter{Confluencia}

En este capítulo estudiaremos el problema de determinar si un sistema
de reescritura es confluente. En la primera sección vamos a demostrar
que este problema es indecidible, sin embargo en las secciones
posteriores, estudiaremos que si el sistema es terminante entonces el
problema es decidible. Por último veremos que ocurre para el caso de
los sistemas que no terminan.

\section{Estudio sobre el problema de decisión}
  
En esta sección veremos la indecibilidad para comprobar si un sistema
es confluente mediante el siguiente resultado,

\begin{teor}
  El problema de decidir si un sistema de reescritura finito $R$ es
  confluente, es indecidible.
\end{teor}

\begin{demo}
  El objetivo de esta demostración es reducir el problema de las
  palabras básicas para $E$ (que sabemos que es indecidible), a un
  sistema de reescritura de términos.

  Sea un conjunto de identidades $E$ tal que $\Var(l) = \Var(r)$ para
  todo $l \approx r \in E$. Sea $R := E \cup E^{-1}$, entonces al
  tener $\rightarrow_R = \leftrightarrow_E$ es confluente. Además $R$
  es un sistema de reescritura (por $\Var(l) = \Var(r)$).

  Dados dos términos básicos $s$ y $t$, y una constante $a$, vamos a
  probar que $R_{st} := R \cup \{ a \rightarrow s, a \rightarrow t \}$
  es confluente syss $s \approx_E t$.

  \begin{itemize}
  \item $(\Rightarrow)$ Si $R_{st}$ es confluente, entonces ni $s$, ni
    $t$ poseen una constante $a$, luego $s \downarrow_{R_{st}} t$. Esto
    significa que las reglas $a \rightarrow s, a \rightarrow t$ no
    pueden ser usadas. Por tanto $\rightarrow_R = \leftrightarrow_E$ y
    $s \approx_E t$.

  \item $(\Rightarrow)$ Vamos a probar que para términos $u, v$, si
    $u \rightarrow_{R_{st}} v$ entonces $v^t \xrightarrow{*}_R u^t$,
    donde $u^t$ denota el resultado de sustituir las constantes $a$ en
    $u$ por $t$.

    Supongamos que $u \rightarrow_{R{_st}} v$. Distinguimos que reglas
    se usan.
    \begin{itemize}
    \item Si usamos $u \rightarrow_R v$, reemplazamos $a$ por $t$ para
      conseguir $u^t \rightarrow_R v^t$ y por tanto
      $v^t \rightarrow_R u^t$ al ser $R$ simétrico.
    \item Si usamos $a \rightarrow s$, entonces $u|_p = a$ y
      $v = u[s]_p$ para alguna posición $p$. Como $s \approx_E t$
      entonces $s \xrightarrow{*}_R t$. Obtenemos
      $v \xrightarrow{*}_R u[t]_p$ que conlleva a
      $v^t \xrightarrow{*}_R (u[t]_p)^t$. Pero $(u[t]_p)^t =
      u^t[t^t]_p = u^t[t]_p = u^t$.
    \item Si usamos $a \rightarrow t$ en la posición $p$, entonces
      $v^t = (u[t]_p)^t = u^t \xrightarrow{*}_R u^t$.
    \end{itemize}
    Por tanto, si $u \xrightarrow{*}_{R_{st}} u_i$, para $i= 1,2$,
    entonces $u_i \xrightarrow{*}_{R_{st}} u_i^t \xrightarrow{*}_R u^t$
    y obtenemos $u_1 \downarrow_{R_{st}}$
  \end{itemize}
\end{demo}

\section{Pares críticos}

En esta sección estudiaremos la decibilidad para sistemas de
reescritura finitos que sean localmente confluentes.

%%%% DIBUJOS SOBRE EL RAZONAMIENTO DE LA DEF DE LOS PARES CRITICOS

\begin{defi}
  Sea $l_i \rightarrow r_i$, $i = 1,2$ dos reglas cuyas variables han
  sido renombradas tal que
  $\Var(l_1,r_1) \cap \Var(l_2,r_2) = \emptyset$. Sea
  $p \in \Pos(l_1)$ tal que $l_1|_p$ no es una variable, y $\theta$ un
  umg de $l_1|_p =^? l_2$. Esto determinará el par crítico $\langle
  \theta r_1, (\theta l_1) [\theta r_2 ]_p \rangle$.
  % DIBUJO DEL ARBOL
  Si dos reglas dan lugar a un par crítico, diremos que se solapan.
\end{defi}







%%%%%%%%%%%%%%%%%%%%%%%%%%%%%%%%%%%%%%%%%%%%%%%%%%%
%%Apéndices
%%%%%%%%%%%%%%%%%%%%%%%%%%%%%%%%%%%%%%%%%%%%%%%%%%%

\clearpage
\addappheadtotoc
\appendix


%%% Local Variables:
%%% mode: latex
%%% TeX-master: "SRT_en_Haskell"
%%% End:
